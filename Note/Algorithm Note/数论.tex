\documentclass[UTF8]{ctexart}
	\title{数论笔记}
	\author{st.krwlng}
	\usepackage{amsmath}
	\usepackage{amssymb}
	\usepackage{amsthm}
\begin{document}
	\maketitle
	\subsection{本原勾股数}
	\paragraph{定义} 任意正整数a, b, c若满足$a^2+b^2=c^2$且$\gcd(a,b)=\gcd(b,c)=\gcd(a,c)=1$,则称a, b, c为一组本源勾股数。
	\paragraph{观测1} 若正整数a, b, c满足$a^2+b^2=c^2$,则$\gcd(a,b)=\gcd(b,c)=\gcd(a,c)$。
	\paragraph{证明}
	\paragraph{1)}推导:$\gcd(a,b){\leq}\gcd(a,c)$ 
	\begin{equation}\label{abac_ceq}
		\begin{aligned}
			c&=\sqrt{a^2+b^2}\\
			 &=\sqrt{(\gcd(a,b)*x)^2+(\gcd(a,b)*y)^2}\\
			 &=\gcd(a,b)*\sqrt{x^2+y^2}\\
			 &=\gcd(a,b)*z
		\end{aligned}
	\end{equation}
	\paragraph{}由公式\ref{abac_ceq}可得
	\begin{equation}\label{abac_gcd_ac}
		\begin{aligned}
			\gcd(a,c)&=\gcd(\gcd(a,b)*x,\gcd(a,b)*z)\\
					&=\gcd(a,b)*\gcd(x,z)\\
					&{\geq}\gcd(a,b)
		\end{aligned}
	\end{equation}

	\paragraph{2)}推导:$\gcd(a,b){\geq}\gcd(a,c)$
	\begin{equation}\label{acab_beq}
		\begin{aligned}
			b&=\sqrt{c^2-a^2}\\
			 &=\sqrt{(\gcd(a,c)*x)^2-(\gcd(a,c)*y)^2}\\
			 &=\gcd(a,c)*\sqrt{x^2-y^2}\\
			 &=\gcd(a,c)*z
		\end{aligned}
	\end{equation}
	\paragraph{}由公式\ref{acab_beq}可得
	\begin{equation}\label{acab_gcd_ab}
		\begin{aligned}
			\gcd(a,b)&=\gcd(\gcd(a,c)*y,\gcd(a,c)*z)\\
					&=\gcd(a,c)*\gcd(y,z)\\
					&{\geq}\gcd(a,c)
		\end{aligned}
	\end{equation}

	\paragraph{3)}综上所述,由公式\ref{abac_gcd_ac}和公式\ref{acab_gcd_ab}可得,$\gcd(a,b)=\gcd(a,c)$成立。

	\paragraph{4)}同理可得,$\gcd(a,c)=\gcd(b,c)$且$\gcd(a,b)=\gcd(b,c)$,因此$\gcd(a,b)=\gcd(b,c)=\gcd(a,c)$成立。\qed

	\paragraph{观测2} 若正整数a, b, c满足$a^2+b^2=c^2$且$\gcd(a,b)=1$,则c一定是奇数且a和b其中一个是偶数。
	\paragraph{证明} 若c为偶数,a, b必须同为奇数(否则根据观测1有$\gcd(a,b){\geq}2$),又有$c^2=a^2+b^2=(a+b)^2-2ab$,其中$(a+b)^2\&3=0$且$2ab\&3=2$。因此$c^2\&3=((a+b)^2-2ab)\&3=2$,即$c^2$有且仅有一个2的因子,因此其平方根的结果不是整数,构成矛盾。\qed
	\paragraph{定理1} 正整数$a$, $b$, $c$是本原勾股数(其中$a$是奇数,$b$是偶数),当且仅当存在互质的正奇数对$s$, $t$(其中$s>t$),满足
	\begin{align}
		\label{eqa}a&=st\\
		\label{eqb}b&=\frac{s^2-t^2}{2}\\
		\label{eqc}c&=\frac{s^2+t^2}{2}
	\end{align}
	\paragraph{证明}
		\subparagraph{引理1} 若$s$, $t$为互质的正奇数(其中$s>t$),则有$s$, $t$, $\frac{s+t}{2}$, $\frac{s-t}{2}$相互互质。
		\subparagraph{证明}
		\subparagraph{1)}推导:$s$与$\frac{s+t}{2}$互质
		\[
			\begin{aligned}
				\gcd(s+t,s)&=\gcd(s,(s+t)-s)\\
				           &=\gcd(s,t)\\
						   &=1
			\end{aligned}
		\]
		\subparagraph{}因此有$1{\leq}\gcd(s,\frac{s+t}{2}){\leq}\gcd(s+t,s)=1{\Rightarrow}\gcd(s,\frac{s+t}{2})=1$。
		\subparagraph{2)}同理可证,$s$与$\frac{s-t}{2}$,$t$与$\frac{s+t}{2}$,$t$与$\frac{s-t}{2}$互质。
		\subparagraph{3)}推导:$\frac{s+t}{2}$与$\frac{s-t}{2}$互质
		\[
			\begin{aligned}
				\gcd(\frac{s+t}{2},\frac{s-t}{2})&=gcd(\frac{(s+t)}{2}+\frac{(s-t)}{2},\frac{s-t)}{2}\\
				                                 &=\gcd(s,\frac{s-t}{2})\\
							                     &=1
			\end{aligned}
		\]\qed
		
		\subparagraph{引理2} 若$\sqrt{st}$为正整数且$s$和$t$互质,则$\sqrt{s}$和$\sqrt{t}$均为正整数。
		\subparagraph{证明} 假如$\sqrt{s}$不是正整数,则s一定包含正奇数个某质因子x。根据$\sqrt{st}$为正整数,可推测t一定也包含正奇数个质因子x,这与$s$和$t$互质矛盾。同理可证,$\sqrt{t}$也一定是正整数。\qed
	
	\paragraph{1)} 推导:$\Leftarrow$
	\begin{equation}\label{abc_equation}
		\begin{aligned}
			a^2+b^2&=s^2t^2+\frac{s^4+t^4-2s^2t^2}{4}\\
			       &=\frac{s^4+t^4+2s^2t^2}{4}\\
				   &=c^2
		\end{aligned}
	\end{equation}
	\begin{equation}\label{ab_gcd}
		\begin{aligned}
			\gcd(a,b)&=\gcd(st,\frac{s^2-t^2}{2})\\
			         &{\leq}\gcd(st,s^2-t^2)\\
					 &=\gcd(st,(s+t)(s-t))\\
					 &{\leq}\gcd(s,s+t)*\gcd(s,s-t)*\gcd(t,s+t)*\gcd(t,s-t)\\
					 &=\gcd(s,t)^4\\
					 &=1
		\end{aligned}
	\end{equation}
	\begin{equation}\label{ac_gcd}
		\begin{aligned}
			\gcd(a,c)&=\gcd(st,\frac{s^2+t^2}{2})\\
			         &=\gcd(st,\frac{(s+t)^2}{2}-st)\\
					 &=\gcd(st,\frac{(s+t)^2}{2})\\
					 &{\leq}\gcd(st,(s+t)^2)\\
					 &=\gcd(st,(s+t)(s+t))\\
					 &{\leq}\gcd(s,s+t)*\gcd(s,s+t)*\gcd(t,s+t)*\gcd(t,s+t)\\
					 &=\gcd(s,t)^4\\
					 &=1
		\end{aligned}
	\end{equation}
	\paragraph{} 由于$s$,$t$为互质的正奇数,因此$s^2$,$t^2$同样为互质的正奇数。由引理1可得到:
	\begin{equation}\label{bc_gcd}
		\begin{aligned}
			\gcd(b,c)=\gcd(\frac{s^2-t^2}{2},\frac{s^2+t^2}{2})=1
		\end{aligned}
	\end{equation}
	\paragraph{} 公式\ref{abc_equation}、\ref{ab_gcd}、\ref{ac_gcd}、\ref{bc_gcd}刚好构成了本原勾股数的性质,因此必要性成立。

	\paragraph{2)} 推导:$\Rightarrow$
	\paragraph{} 构造$s=\sqrt{c+b}$,$t=\sqrt{c-b}$。将$s$、$t$带入公式\ref{eqa}、\ref{eqb}、\ref{eqc}显然成立,因此下面只需要证明$s$、$t$都是正奇数且互质,即可以证明定理的充分性。
	\paragraph{} 由于b为偶数,c为奇数(由观测2得到),因此$\gcd(b+c,2)=1$。再根据$\gcd(b,c)=1$,可推出:
	\begin{equation}\label{sqrs_sqrt_gcd}
		\begin{aligned}
			\gcd(s^2,t^2)&=\gcd(c+b,c-b)\\
			             &=\gcd(b+c,2c)\\
			             &{\leq}\gcd(b+c,2)*\gcd(b+c,c)\\
						 &=\gcd(b+c,2)*\gcd(b,c)\\
						 &=1
		\end{aligned}
	\end{equation}
	\paragraph{} 由$a=\sqrt{s^2t^2}$为正奇数,根据公式\ref{sqrs_sqrt_gcd}和引理2可证,$s$和$t$都是正整数。再根据b和c的奇偶性可判断,$s$和$t$都是正奇数。
	\paragraph{} 最后,根据$\gcd(s,t){\leq}\gcd(s^2,t^2)=1$可知$s$和$t$互质。\qed	
\end{document}
