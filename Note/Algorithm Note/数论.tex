\documentclass[UTF8]{ctexart}
	\title{数论笔记}
	\author{st.krwlng}
	\usepackage{amsmath}
	\usepackage{amssymb}
	\usepackage{amsthm}
\begin{document}
	\maketitle
	\subsection{本原勾股数}
	\paragraph{定义} 任意正整数a, b, c若满足$a^2+b^2=c^2$且$\gcd(a,b)=\gcd(b,c)=\gcd(a,c)=1$,则称a, b, c为一组本源勾股数。
	\paragraph{观测1} 若正整数a, b, c满足$a^2+b^2=c^2$,则$\gcd(a,b)=\gcd(b,c)=\gcd(a,c)$。
	\paragraph{证明}
	\paragraph{1)}推导:$\gcd(a,b){\leq}\gcd(a,c)$ 
	\begin{equation}\label{abac_ceq}
		\begin{aligned}
			c&=\sqrt{a^2+b^2}\\
			 &=\sqrt{(\gcd(a,b)*x)^2+(\gcd(a,b)*y)^2}\\
			 &=\gcd(a,b)*\sqrt{x^2+y^2}\\
			 &=\gcd(a,b)*z
		\end{aligned}
	\end{equation}
	\paragraph{}由公式\ref{abac_ceq}可得
	\begin{equation}\label{abac_gcd_ac}
		\begin{aligned}
			\gcd(a,c)&=\gcd(\gcd(a,b)*x,\gcd(a,b)*z)\\
					&=\gcd(a,b)*\gcd(x,z)\\
					&{\geq}\gcd(a,b)
		\end{aligned}
	\end{equation}

	\paragraph{2)}推导:$\gcd(a,b){\geq}\gcd(a,c)$
	\begin{equation}\label{acab_beq}
		\begin{aligned}
			b&=\sqrt{c^2-a^2}\\
			 &=\sqrt{(\gcd(a,c)*x)^2-(\gcd(a,c)*y)^2}\\
			 &=\gcd(a,c)*\sqrt{x^2-y^2}\\
			 &=\gcd(a,c)*z
		\end{aligned}
	\end{equation}
	\paragraph{}由公式\ref{acab_beq}可得
	\begin{equation}\label{acab_gcd_ab}
		\begin{aligned}
			\gcd(a,b)&=\gcd(\gcd(a,c)*y,\gcd(a,c)*z)\\
					&=\gcd(a,c)*\gcd(y,z)\\
					&{\geq}\gcd(a,c)
		\end{aligned}
	\end{equation}

	\paragraph{3)}综上所述,由公式\ref{abac_gcd_ac}和公式\ref{acab_gcd_ab}可得,$\gcd(a,b)=\gcd(a,c)$成立。

	\paragraph{4)}同理可得,$\gcd(a,c)=\gcd(b,c)$且$\gcd(a,b)=\gcd(b,c)$,因此$\gcd(a,b)=\gcd(b,c)=\gcd(a,c)$成立。\qed

	\paragraph{观测2} 若正整数a, b, c满足$a^2+b^2=c^2$且$\gcd(a,b)=1$,则c一定是奇数且a和b其中一个是偶数。
	\paragraph{证明} 若c为偶数,a, b必须同为奇数(否则根据观测1有$\gcd(a,b){\geq}2$),又有$c^2=a^2+b^2=(a+b)^2-2ab$,其中$(a+b)^2\&3=0$且$2ab\&3=2$。因此$c^2\&3=((a+b)^2-2ab)\&3=2$,即$c^2$有且仅有一个2的因子,因此其平方根的结果不是整数,构成矛盾。\qed
	\paragraph{定理1} 正整数$a$, $b$, $c$是本原勾股数(其中$a$是奇数,$b$是偶数),当且仅当存在互质的正奇数对$s$, $t$(其中$s>t$),满足
	\begin{align}
		\label{eqa}a&=st\\
		\label{eqb}b&=\frac{s^2-t^2}{2}\\
		\label{eqc}c&=\frac{s^2+t^2}{2}
	\end{align}
	\paragraph{证明}
		\subparagraph{引理1} 若$s$, $t$为互质的正奇数(其中$s>t$),则有$s$, $t$, $\frac{s+t}{2}$, $\frac{s-t}{2}$相互互质。
		\subparagraph{证明}
		\subparagraph{1)}推导:$s$与$\frac{s+t}{2}$互质
		\[
			\begin{aligned}
				\gcd(s+t,s)&=\gcd(s,(s+t)-s)\\
				           &=\gcd(s,t)\\
						   &=1
			\end{aligned}
		\]
		\subparagraph{}因此有$1{\leq}\gcd(s,\frac{s+t}{2}){\leq}\gcd(s+t,s)=1{\Rightarrow}\gcd(s,\frac{s+t}{2})=1$。
		\subparagraph{2)}同理可证,$s$与$\frac{s-t}{2}$,$t$与$\frac{s+t}{2}$,$t$与$\frac{s-t}{2}$互质。
		\subparagraph{3)}推导:$\frac{s+t}{2}$与$\frac{s-t}{2}$互质
		\[
			\begin{aligned}
				\gcd(\frac{s+t}{2},\frac{s-t}{2})&=gcd(\frac{(s+t)}{2}+\frac{(s-t)}{2},\frac{s-t)}{2}\\
				                                 &=\gcd(s,\frac{s-t}{2})\\
							                     &=1
			\end{aligned}
		\]\qed
		
		\subparagraph{引理2} 若$\sqrt{st}$为正整数且$s$和$t$互质,则$\sqrt{s}$和$\sqrt{t}$均为正整数。
		\subparagraph{证明} 假如$\sqrt{s}$不是正整数,则s一定包含正奇数个某质因子x。根据$\sqrt{st}$为正整数,可推测t一定也包含正奇数个质因子x,这与$s$和$t$互质矛盾。同理可证,$\sqrt{t}$也一定是正整数。\qed
	
	\paragraph{1)} 推导:$\Leftarrow$
	\begin{equation}\label{abc_equation}
		\begin{aligned}
			a^2+b^2&=s^2t^2+\frac{s^4+t^4-2s^2t^2}{4}\\
			       &=\frac{s^4+t^4+2s^2t^2}{4}\\
				   &=c^2
		\end{aligned}
	\end{equation}
	\begin{equation}\label{ab_gcd}
		\begin{aligned}
			\gcd(a,b)&=\gcd(st,\frac{s^2-t^2}{2})\\
			         &{\leq}\gcd(st,s^2-t^2)\\
					 &=\gcd(st,(s+t)(s-t))\\
					 &{\leq}\gcd(s,s+t)*\gcd(s,s-t)*\gcd(t,s+t)*\gcd(t,s-t)\\
					 &=\gcd(s,t)^4\\
					 &=1
		\end{aligned}
	\end{equation}
	\begin{equation}\label{ac_gcd}
		\begin{aligned}
			\gcd(a,c)&=\gcd(st,\frac{s^2+t^2}{2})\\
			         &=\gcd(st,\frac{(s+t)^2}{2}-st)\\
					 &=\gcd(st,\frac{(s+t)^2}{2})\\
					 &{\leq}\gcd(st,(s+t)^2)\\
					 &=\gcd(st,(s+t)(s+t))\\
					 &{\leq}\gcd(s,s+t)*\gcd(s,s+t)*\gcd(t,s+t)*\gcd(t,s+t)\\
					 &=\gcd(s,t)^4\\
					 &=1
		\end{aligned}
	\end{equation}
	\paragraph{} 由于$s$,$t$为互质的正奇数,因此$s^2$,$t^2$同样为互质的正奇数。由引理1可得到:
	\begin{equation}\label{bc_gcd}
		\begin{aligned}
			\gcd(b,c)=\gcd(\frac{s^2-t^2}{2},\frac{s^2+t^2}{2})=1
		\end{aligned}
	\end{equation}
	\paragraph{} 公式\ref{abc_equation}、\ref{ab_gcd}、\ref{ac_gcd}、\ref{bc_gcd}刚好构成了本原勾股数的性质,因此必要性成立。

	\paragraph{2)} 推导:$\Rightarrow$
	\paragraph{} 构造$s=\sqrt{c+b}$,$t=\sqrt{c-b}$。将$s$、$t$带入公式\ref{eqa}、\ref{eqb}、\ref{eqc}显然成立,因此下面只需要证明$s$、$t$都是正奇数且互质,即可以证明定理的充分性。
	\paragraph{} 由于b为偶数,c为奇数(由观测2得到),因此$\gcd(b+c,2)=1$。再根据$\gcd(b,c)=1$,可推出:
	\begin{equation}\label{sqrs_sqrt_gcd}
		\begin{aligned}
			\gcd(s^2,t^2)&=\gcd(c+b,c-b)\\
			             &=\gcd(b+c,2c)\\
			             &{\leq}\gcd(b+c,2)*\gcd(b+c,c)\\
						 &=\gcd(b+c,2)*\gcd(b,c)\\
						 &=1
		\end{aligned}
	\end{equation}
	\paragraph{} 由$a=\sqrt{s^2t^2}$为正奇数,根据公式\ref{sqrs_sqrt_gcd}和引理2可证,$s$和$t$都是正整数。再根据b和c的奇偶性可判断,$s$和$t$都是正奇数。
	\paragraph{} 最后,根据$\gcd(s,t){\leq}\gcd(s^2,t^2)=1$可知$s$和$t$互质。\qed	
    \subsection{最大公约数}
    \paragraph{定义} 两个正整数a, b的最大公约数是a和b都能整除的最大的正整数,用符号表示为$\gcd(a,b)$。多个正整数的最大公约数是能被这些正整数整除的最大的正整数。
    \paragraph{观测1} 最大公约数运算满足交换律、结合律和分配律。即
    \begin{equation}\label{gcd_prop}
        \begin{aligned}
            \gcd(a,b)&=\gcd(b,a)\\
            \gcd(\gcd(a,b),c)&=\gcd(a,\gcd(b,c)) (=\gcd(a,b,c))\\
            \gcd(a*t,b*t)&=t*gcd(a,b)
        \end{aligned}
    \end{equation}
    \paragraph{证明} 证明略,都可使用反证法证明。\qed
    \paragraph{观测2} 取$p=a/\gcd(a,b), q=b/\gcd(a,b)$,有$\gcd(p,q)=1$。
    \paragraph{证明} 若$\gcd(p,q)>1$,由\ref{gcd_prop}有$\gcd(a,b)=\gcd(p*\gcd(a,b),q*\gcd(a,b))=\gcd(a,b)*\gcd(p,q)>\gcd(a,b)$,故矛盾。\qed
    \paragraph{观测3} 若正整数b,c互质,则有$\gcd(a*c,b)=\gcd(a,b)$。
    \paragraph{证明}
    \paragraph{1)} $\gcd(a*c,b)\leq\gcd(a,b)*\gcd(c,b)=\gcd(a,b)$
    \paragraph{2)} $\gcd(a*c,b)\geq\gcd(a,b)$
    \paragraph{3)} 上述推导的不等式可以在质因数分解的部分利用集合的交集与并集的性质证明。综上所述,$\gcd(a*c,b)=\gcd(a,b)$
    \paragraph{定理1(欧几里得定理)} 对任意正整数有$\gcd(a,b)=\gcd(a+b*t,b)$成立,其中t为任意整数,使得a+b*t为正整数。
    \paragraph{证明}
    \paragraph{} 取$p=a/\gcd(a,b), q=b/\gcd(a,b)$
    \paragraph{1)} $\gcd(a+b*t,b)=\gcd(\gcd(a,b)*(p+qt),\gcd(a,b)*q)=\gcd(a,b)*gcd(p+qt,q)\geq\gcd(a,b)$
    \paragraph{2)} $\gcd(a,b)=\gcd(a+b*t+b*(-t),b)\geq\gcd(a+b*t,b)$
    \paragraph{3)} 综上所述,$\gcd(a,b)=\gcd(a+b*t,b)$成立。 \qed
    \subsection{线性同余方程}
    \paragraph{定义} 对任意正整数a、b和非负且小于b的整数c,求解一次整数x的方程ax mod b = c为线性同余方程。线性同余方程还可被写作ax+by=c,在这种表示方式中c没有取值范围限制,整数x是唯一的未知数。
    \paragraph{观测1} 任意正整数a、b,线性运算ax+by的值域是所有能被$\gcd(a,b)$整除的整数。
    \paragraph{证明}
    \subparagraph{引理1} 若a、b互质,则数列$A_i=a*i\mod b(i\geq0)$的前b个元素构成0至b-1的一个排列。
    \subparagraph{证明} 若数列的前b个元素中存在相同的数,取相同值的小标分别为i和j,即$A_i=A_j \And i<b \And j<b \And j>i$。有
    \begin{equation}
        \begin{aligned}
            &&&A_j-A_i=a*j\mod b-a*i\mod b=a*(j-i)\mod b=0\\
            &\Rightarrow &&\gcd(a*(j-i),b)\\
            &&=&\gcd(j-i,b)*\gcd(a*((j-i)/\gcd(j-i,b)),b/\gcd(j-i,b))\\
            &&=&\gcd(j-i,b)*\gcd(a,b/\gcd(j-i,b))\\
            &&=&b\\
            &\Rightarrow &&\gcd(a,b/\gcd(j-i,b))\\
            &&=&b/\gcd(j-i,b)\\
            &&\geq&b/\min(j-i,b)\\
            &&=&b/(j-i)\\
            &&>&b/b\\
            &&=&1\\
            &\Rightarrow &&\gcd(a,b)>1
        \end{aligned}
    \end{equation}
    \subparagraph{} 上式使用了最大公约数部分的若干观测结论,推导结果与a、b互质矛盾。\qed
    \paragraph{1)} 设x、y为任意整数,取$p=a/\gcd(a,b),q=b/\gcd(a,b)$,有$ax+by=gcd(a,b)*(px+qy)$,故一定能被$\gcd(a,b)$整除。
    \paragraph{2)} 由引理1可得推论,存在$x'$与$y'$使得$ax'+by'=\gcd(a,b)$:取$p=a/\gcd(a,b),q=b/\gcd(a,b)$,可知存在$0\leq x'\leq q-1$使得$p*x'\mod q=1$,再取$y'=-(p*x'\div y')$,则有$px'+qy'=1$,则有$ax'+by'=\gcd(a,b)*(px'+qy')=\gcd(a,b)$。对任意能被$\gcd(a,b)$整除的整数$t*\gcd(a,b)$,取$x=t*x'$,$y=t*y'$,则有$ax+by=t*(ax'+by')=t*\gcd(a,b)$。
    \paragraph{3)} 综上所述,整数线性运算ax+by的值域是所有能被$\gcd(a,b)$整除的整数。
    \paragraph{线性同余方程解法(扩展欧几里得定理)} 线性同余方程$ax+by=\gcd(a,b)$的解可以在使用欧几里得定理求gcd(a,b)的递归过程中得到,具体为:
    \paragraph{1)} 若$a\mod b=0$,则$\gcd(a,b)=b\Rightarrow x=0 \And y=1$。
    \paragraph{2)} 若已求得$by'+cx'=\gcd(b,c)=\gcd(a,b)$,其中$c=a+b*t$,即$a=c-b*t$,则有
    \begin{equation}
        \begin{aligned}
            && &ax'+by'=(c-b*t)x'+by'=cx'+by'-b*t*x'=gcd(a,b)-b*t*x'\\
            &\Rightarrow& &ax'+by'+b*t*x'=ax'+b*(y'+t*x')=\gcd(a,b)\\
            &\Rightarrow& &x=x' \And y=y'+t*x'
        \end{aligned}
    \end{equation}
    \subsection{逆元}
    \paragraph{定义} 对于整数a,b以及任意能被a整除的整数x,若存在整数c,使得$x/a \mod b=x*c \mod b$,则称c是a关于域b的逆元。
    \paragraph{逆元的意义} 取模运算相对于加法、减法和乘法都满足分配律,即$(a+b)\mod N=(a\mod N+b\mod N)\mod N$、$(a-b)\mod N=(a\mod N-b\mod N)\mod N$、$(a*b)\mod N=(a\mod N)*(b\mod N)\mod N$,然而除法不满足分配律,即$a/b\mod N\neq (a\mod N/b)\mod N$。但若存在b关于域N的逆元c,则可将除法转换为乘法,即$a/b\mod N=a*c\mod N=(a\mod N*c)\mod N$(必须以a能被b整除为前提)。
    \paragraph{定理1} 若a,b互质,则a关于域b的逆元是线性同余方程的解$ax\mod b=1$。
    \paragraph{证明} 设t是任意能被a整除的整数,s=t/a,则有
    \begin{equation}
        \begin{aligned}
            & & &(t/a)\mod b\\
            &=& &s\mod b\\
            &=& &s*1\mod b\\
            &=& &s*(ax)\mod b\\
            &=& &t*x\mod b
        \end{aligned}
    \end{equation}
    满足逆元的定义。\qed
\end{document}
